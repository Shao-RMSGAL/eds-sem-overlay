\documentclass[10pt]{exam}
\usepackage{amsthm}
\usepackage{libertine}
\usepackage[utf8]{inputenc}
\usepackage[T1]{fontenc}
\usepackage[margin=0.25in, includefoot]{geometry}
\usepackage{amsmath,amssymb}
\usepackage{multicol}
\usepackage[shortlabels]{enumitem}
\usepackage{cancel}
\usepackage{listings}
\usepackage{tikz}
\usepackage{float}
\usepackage{mathtools}
\usepackage{wrapfig, lipsum}
\usepackage{chemformula}
\usepackage{empheq}
\usepackage[hypcap=false]{caption}
\usepackage{subcaption}
\usepackage[colorlinks=true, linkcolor=blue, urlcolor=blue]{hyperref}
% \usepackage[normalem]{ulem}
\usepackage{graphicx}
\usepackage{bm}
\usepackage{accents}
\usepackage[T1]{fontenc}
\usepackage[font=small,labelfont=bf,tableposition=top]{caption}
\usepackage{siunitx}
\usepackage{minted}
\usepackage{etoolbox}
\usepackage{multirow}
\usepackage[version=4]{mhchem}
\usepackage{pgfplots}
\usepackage{svg}
\usepackage{microtype}
\usepackage{blindtext}
\usepackage{booktabs}
\usepackage{booktabs}
\usepackage{pdfpages}
\usepackage{nicematrix}
\usepackage{cleveref}
\pgfplotsset{compat=1.18}


\AtBeginEnvironment{minted}{
\fontsize{8}{10}\selectfont}
\newcommand{\fahrenheit}{^\circ{F}}
\newcommand*\widefbox[1]{\fbox{\hspace{2em}#1\hspace{2em}}}
\newcommand{\msout}[1]{
    \text{\sout{$#1$}}
}
\usetikzlibrary{shapes}
\DeclareSIUnit\year{y}
\sisetup{
group-digits=integer,
group-minimum-digits=3,
group-separator={,}
}
% \newcommand{\class}{NUEN 604 - 600}
% \newcommand{\examnum}{Homework 1}
% \newcommand{\examdate}{September 11\textsuperscript{th}}
% \newcommand{\timelimit}{}
% \newcommand{\laplace}{\mathcal{L}}

\newcommand{\boxedanswer}[3][0.93]{
    \fbox{
        \begin{minipage}{#1\textwidth}
            #2
        \end{minipage}
        }
    }

\begin{document}

\section{Method overview}

To determine the intergranular versus intragranular attack, a software was 
written in Python
which analyzes EDS images for areas of high Cr concentration and compares 
the data to SEM images. Images can be represented as matrices of pixels, 
and each pixel in the EDS data can be represented by three
numbers, red, white, and blue. The range of these numbers range from 0 to 255,
where the number represents the presence of each color. Because the EDS data
is monochrome, the image was grayscaled, which means each pixel was set
to the average RGB value. \par

The SEM was already grayscale, so no pre-processing was done for the
SEM images. Both images were then converted to black and white, where light
pixels were converted to white, and dark pixels were converted to black. A 
manually chosen threshold was used to determine the pixel value threshold
which differentiates white pixels from black pixels. The software provided
a black-and-white image next to the original image to allow the user to 
select a threshold which adequately represented voids in the SEM image, 
as well as Cr depletion in EDS images. \par

Each pixel in both images were then counted row-by-row, and a dataset was built
according to \cref{eq:counting}.

\begin{equation}
    c_{i} = \sum_{j}^W p_{i,j} 
    \label{eq:counting}
\end{equation}

where $c_{i}$ is the count of white pixels in row $i$ in the pixel matrix, 
$j$ is the column of the matrix, $W$ is the width of the pixel matrix in the 
pixel matrix, and $p_{i,j}$ is the value of the pixel in row $i$ and column $j$
(1 for black and 0 for white).
The corresponding depth of each row was found by converting pixels to 
depth using the scale bars on the SEM and EDS images, and the number of black
pixels per row as a function of depth was found. For the SEM images, this 
represented the voids in the material, and for EDS images, this represented
the chromium depletion in the material. \par

To determine the intergranular corrosion as a function of depth, the 
SEM pixel matrix was compared pixel-by-pixel to the EDS pixel matrix using
equation \ref{eq:compare}. \par

\begin{equation}
    o_{i,j} = \begin{cases}
        1 & \text{ if }  p^{\text{EDS}}_{i,j} = 1 \text{ and } p^{\text{SEM}}_{i,j} = 1 \\
        0 & \text{ if otherwise}   \\
    \end{cases}
    \label{eq:compare}
\end{equation}

Where $o_{i,j}$ is the overlaid pixel value of the EDS and SEM images at row
$i$ and column $j$, $p^{\text{EDS}}_{i,j}$ is the value of the EDS pixel matrix
 and $p^{\text{SEM}}_{i,j}$ is the value of the SEM pixel matrix. This data was
 then counted using \cref{eq:counting} to get the intergranular corrosion.
 Intragranular corrosion was determined using \cref{eq:compare2}.

\begin{equation}
    o_{i,j} = \begin{cases}
        1 & \text{ if }  p^{\text{EDS}}_{i,j} = 1 \text{ and } p^{\text{SEM}}_{i,j} = 0 \\
        0 & \text{ if otherwise}   \\
    \end{cases}
    \label{eq:compare2}
\end{equation}

\end{document}
